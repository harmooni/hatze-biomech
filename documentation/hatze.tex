\documentclass[a4paper]{article}
\usepackage{biblatex}
\bibliography{ARK}
\usepackage{hyperref}
\begin{document}

\title{Hatze's legacy}
\author{Will Robertson}
\maketitle

\section{Introduction}

Herbert Hatze (1937-2002) had a major on the field of biomechanics.\footnote{\url{http://biomch-l.isbweb.org/threads/12847-Obituary-Professor-Herbert-Hatze}}
While certain aspects of his work have now become out-dated, there is much to his work that is still as fresh today as when it was first published in the 1970s--1980s.
This document serves to outline Hatze's contributions.

The first aspect of his work that is covered in this document is Hatze's work on body segment parameter modelling.
Hatze's own model \parencite{hatze1979-techreport} is still the most detailed geometric model of an entire humanoid figure; as Hatze himself wrote\footnote{\url{http://biomch-l.isbweb.org/threads/12059-DISCUSSION-FORUM-ON-CONTEMPORARY-ISSUES-IN-BIOMECHANICS}} in late 2001:
\begin{quote}
For any model of the segmental, muscular, articular, or neural subsystem
to be implemented in practice, the values of the respective
subject-specific parameter sets must be available.
Comparatively little
effort has been devoted to this extremly important field of
biomechanical research despite its obvious practical relevance.
\end{quote}

\printbibliography
\end{document}